\documentclass[a4paper,11pt]{jsarticle}

\usepackage{amsmath,amsthm,amsfonts,float,cases,bm,amssymb,amssymb,ascmac,url,color,enumitem}
\usepackage[dvipdfmx]{graphicx}
\usepackage[all,pdf]{xy}%pdfがないと矢印に色がつかない
%\usepackage{makeidx}%これを使ったうえで\index{}を使わないとエラーになる
\usepackage[bbgreekl]{mathbbol}
\usepackage[%,
dvipdfmx,% 欧文ではコメントアウトする
setpagesize=false,%
bookmarks=true,%
bookmarksdepth=tocdepth,%
bookmarksnumbered=true,%
colorlinks=false,%
pdftitle={},%
pdfsubject={},%
pdfauthor={},%
pdfkeywords={}%
]{hyperref}% PDFのしおり機能の日本語文字化けを防ぐ((u)pLaTeXのときのみかく)
\usepackage{pxjahyper}
\DeclareMathSymbol{\bbepsilon}{\mathord}{bbold}{"0F}


%\makeindex%インデックスつかわないときにこれoffにしないとエラー出る

\setlength{\textwidth}{\fullwidth}
\setlength{\textheight}{40\baselineskip}
\addtolength{\textheight}{\topskip}
\setlength{\voffset}{-0.55in}

\theoremstyle{definition}
\newtheorem{thm}{定理}[section]
\newtheorem{prop}[thm]{命題}
\newtheorem{cor}[thm]{系}
\newtheorem{dfn}[thm]{定義}
\newtheorem{lem}[thm]{補題}
\newtheorem{rem}[thm]{注意}
\newtheorem{eg}[thm]{例}

\DeclareMathOperator{\Hom}{\mathrm{Hom}}
\DeclareMathOperator{\id}{\mathrm{id}}
\DeclareMathOperator{\Int}{\mathrm{Int}}

\setcounter{tocdepth}{3}%目次に表示する数字の深さ
\setcounter{section}{-1}

\begin{document}
\date{}
\title{クマの釜山のホームランダービー!公式ガイドブック}

\maketitle

%\begin{abstract}
%\end{abstract}

%\tableofcontents

\section{遊び方}
ダウンロードした\verb|KamayamaGame.zip|を適当な場所に解凍し、フォルダ内の\verb|KamayamaGame.exe|をダブルクリックすることで遊戯を開始できます。

\section{理念}

昔、Yahoo!きっずゲームには「くまのプーさんのホームランダービー!」という、Adobe Flash Playerのサービス終了とともに消えた名作ゲームがありました。その魅力の詳細については省略しますが、子供向けとは思えない高難度であったため、大晦日のお供としてこれのスピードランに挑戦する大人もいました。

これが遊べなくなった今、彼らの大晦日に彩りはありますか?

このゲームは、そんな彼らに、大晦日の色を思い出してもらうべく作られました。

\section{ゲーム開発を始めた経緯}

制作者は、もともとWindowsのデスクトップで実行可能なソフトを作りたいと思っていましたが、プログラミングの知識がほとんどなかったので、その勉強をかねてゲーム開発を開始しました。


\section{ゲーム開発を勉強するときのこだわり}

なるべくWindowsデスクトップアプリ開発およびゲーム開発を低レベルから学びたかったので、用いた技術は、プログラミング言語C++と、Win32APIという、ネイティブなWindowsデスクトップアプリを開発するためにMicrosoftが公開しているC++ライブラリがほとんどです。UnityやUEのようなゲームエンジンに頼らないということです。どこかで「C++とDirectX\footnote{Windowsで動く、低レベルな3Dグラフィックエンジン}があればなんでもできる」というような思想を見たことがありますが、今回の開発思想はこれに大きく影響されています。

しかし、ゲーム開発の勉強という点では、むしろゲームエンジンを使う方がおすすめだと思いました。実際に低レベル技術のみでゲームを開発してみると、ゲームオブジェクトやステート、レンダリング、オーディオを柔軟かつ安全に管理できるようなコード設計を(時にChatGPTに頼りながら)自力で行う必要がありました。これは実質的に、簡易的なゲームエンジンを制作していることになります。今回の開発ではゲーム内で実現したい演出、機能などの実装とは関係ない部分に8割ほどの時間を費やしてしまいました。

他人が作ったゲームエンジンに頼らないゲーム制作を勉強したい場合、まずは有名なゲームエンジンがゲームに含まれる要素(ステート、オブジェクト、オーディオ、ゲームループ、レンダリング等)をどのようなロジックで管理しているかを、ゲームエンジンに頼ったゲーム開発を通して学んでからにするのが良いです。


上記のように、「ネイティブなWindowsデスクトップアプリ開発」と「ゲーム開発」と「ゲームエンジン開発」を同時に勉強しようとするとこれらの学習内容が絡まり、解決するべき問題が初心者にとって複雑になりすぎます。できれば、切り離して勉強するべきです。

\section{これからどうするか}

このゲームは未完成です。もうすこし機能の追加等のアップデートを行う予定です。

\bibliographystyle{alpha}
\bibliography{bib}

%printindex

\end{document}